\chapter{Mapserver}
\section{Instalasi Mapserver}
\subsection{Langkah-langkah instalasi Mapserver}
\begin{enumerate}
    \item Download Aplikasi Map Server atau MS4W, pilih ms4w-3.2.8-setup.exe.
    \item Setelah selesai download, pili file yang terdownload kemudian klik kanan, pilih Run Administrator. Maka akan muncul seperti gambar \ref{instal1}, klik double I Agree.
    \begin{figure}[!htbp]
    \centering
    \includegraphics[scale=0.5]{figures/instal1.jpg}
    \label{instal1}
\end{figure}
    \item Selanjutnya pilih MapServer Itasca Demo Application, dan jangan ubah yang lainnya, kemudian pilih Next > \ref{instal2}.
    \begin{figure}[!htbp]
    \centering
    \includegraphics[scale=0.5]{figures/instal2.jpg}
    \label{instal2}
\end{figure}
    \item Setelah itu pilih penyimpanan nya dan saya akan menyimpannya di C:\ kemudian pilih Next> \ref{instal3}. 
    \begin{figure}[!htbp]
    \centering
    \includegraphics[scale=0.5]{figures/instal3.jpg}
    \label{instal3}
\end{figure}
    \item Masukkan port 82 Pada Apache Port kemudian pilih Install, seperti gambar \ref{instal4}
     \begin{figure}[!htbp]
    \centering
    \includegraphics[scale=0.5]{figures/instal4.jpg}
    \label{instal4}
\end{figure}
    \item Tunggu proses instalisasi nya sepreti gambar \ref{instal5}. 
    \begin{figure}[!htbp]
    \centering
    \includegraphics[scale=0.5]{figures/instal5.jpg}
    \label{instal5}
\end{figure}
    \item Setelah instalasi nya selesai akan muncul seperti ini, kemudian tunggu lagi prosesnya sampai Complete seperti gambar \ref{instal6}.
    \begin{figure}[!htbp]
    \centering
    \includegraphics[scale=0.5]{figures/instal6.jpg}
    \label{instal6}
\end{figure}    
    \item Pada pencarian masukkan localhost:82 dimana 82 merupakan port yang telah di tetapkan pada proses instalasi Map Server nya, seperti pada gambar \ref{instal7}. 
    \begin{figure}[!htbp]
    \centering
    \includegraphics[scale=0.4]{figures/instal7.jpg}
    \label{instal7}
\end{figure}  

\end{enumerate}

\section{Membaca Shapefile dengan PySHP}
\lstinputlisting[language=Python, breaklines=true, caption=koding]{src/1.py}
\par Hasil dari script diatas adalah: 
\begin{figure}[!htbp]
    \centering
    \includegraphics[scale=0.7]{figures/soal1.PNG}
    \label{soal1}
\end{figure}

\section{Soal 2}
\lstinputlisting[language=Python, breaklines=true, caption=koding]{src/2.py}

\par Hasil dari script python diatas adalah:
\begin{figure}[!htbp]
    \centering
    \includegraphics[scale=0.7]{figures/soal2.PNG}
    \label{soal2}
\end{figure}

\section{Soal 3}
\lstinputlisting[language=Python, breaklines=true, caption=koding]{src/3.py}
\par Hasil dari script python soal 3 diatas adalah: 
\begin{figure}[!htbp]
    \centering
    \includegraphics[scale=0.5]{figures/soal3.PNG}
    \label{soal3}
\end{figure}

\section{Soal 4}
\lstinputlisting[language=Python, breaklines=true, caption=koding]{src/4.py}

\par Hasil dari script python soal 4 diatas adalah: 
\begin{figure}[!htbp]
    \centering
    \includegraphics[scale=0.5]{figures/soal4.PNG}
    \label{soal4}
\end{figure}

\section{Soal 5}
\lstinputlisting[language=Python, breaklines=true, caption=koding]{src/5.py}

\par Hasil dari script python soal 5 diatas adalah: 
\begin{figure}[!htbp]
    \centering
    \includegraphics[scale=0.7]{figures/soal5.PNG}
    \label{soal5}
\end{figure}

\section{Soal 6}
\lstinputlisting[language=Python, breaklines=true, caption=koding]{src/6.py}

\par Hasil dari script python soal 6 diatas adalah: 
\begin{figure}[!htbp]
    \centering
    \includegraphics[scale=0.7]{figures/soal6.PNG}
    \label{soal6}
\end{figure}

\section{Soal 7}
\lstinputlisting[language=Python, breaklines=true, caption=koding]{src/7.py}

\par Hasil dari script python soal 7 diatas adalah: 
\begin{figure}[!htbp]
    \centering
    \includegraphics[scale=0.7]{figures/soal7.PNG}
    \label{soal7}
\end{figure}

\section{Soal 8}
\lstinputlisting[language=Python, breaklines=true, caption=koding]{src/8.py}

\par Hasil dari script python soal 8 diatas adalah: 
\begin{figure}[!htbp]
    \centering
    \includegraphics[scale=0.7]{figures/soal8.PNG}
    \label{soal8}
\end{figure}

\section{Soal 9}
\lstinputlisting[language=Python, breaklines=true, caption=koding]{src/9.py}

\par Hasil dari script python soal 9 diatas adalah: 
\begin{figure}[!htbp]
    \centering
    \includegraphics[scale=0.7]{figures/soal9.PNG}
    \label{soal9}
\end{figure}

\section{Soal 10}
\par Hasil dari script python soal 10 diatas adalah: 
\begin{figure}[!htbp]
    \centering
    \includegraphics[scale=0.7]{figures/soal10.PNG}
    \label{soal10}
\end{figure}





 

