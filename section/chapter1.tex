\chapter{Menulis dan Membaca Shapefile dengan PySHP}
\section{Soal 1}
\begin{verbatim}
import shapefile
w=shapefile.Writer()
w.shapeType
w.field("kolom1","C")
w.field("kolom2","C")
w.record("ngek","satu")
w.record("ngok","dua")
w.point(1,1)
w.point(2,2)
w.save("soal1")
\end{verbatim}

\par Hasil dari script diatas adalah: 
\begin{figure}[!htbp]
    \centering
    \includegraphics[scale=0.7]{figures/soal1.PNG}
    \label{soal1}
\end{figure}

\section{Soal 2}
\begin{verbatim}
import shapefile
w=shapefile.Writer(shapeType=1)
w.shapeType
w.field("kolom1","C")
w.field("kolom2","C")
w.record("ngek","satu")
w.record("ngok","dua")
w.point(1,1)
w.point(2,2)
w.save("soal2")
\end{verbatim}

\par Hasil dari script python diatas adalah:
\begin{figure}[!htbp]
    \centering
    \includegraphics[scale=0.7]{figures/soal2.PNG}
    \label{soal2}
\end{figure}

\section{Soal 3}
\begin{verbatim}
import shapefile
w=shapefile.Writer(shapeType=1)
w.shapeType
w.shapeType=3
w.shapeType
w.field("kolom1","C")
w.field("kolom2","C")
w.record("ngek","satu")
w.record("ngok","dua")
w.point(1,1)
w.point(2,2)
w.save("soal3")
\end{verbatim}

\par Hasil dari script python soal 3 diatas adalah: 
\begin{figure}[!htbp]
    \centering
    \includegraphics[scale=0.5]{figures/soal3.PNG}
    \label{soal3}
\end{figure}

\section{Soal 4}
\begin{verbatim}
import shapefile
w=shapefile.Writer(shapefile.POINTM)
w.shapeType
w.field("kolom1","C")
w.field("kolom2","C")
w.record("ngek","satu")
w.record("ngok","dua")
w.point(1,1)
w.point(2,2)
w.save("soal4")
\end{verbatim}

\par Hasil dari script python soal 4 diatas adalah: 
\begin{figure}[!htbp]
    \centering
    \includegraphics[scale=0.5]{figures/soal4.PNG}
    \label{soal4}
\end{figure}

\section{Soal 5}
\begin{verbatim}
import shapefile
w=shapefile.Writer()
w.shapeType
w.field("kolom1","C")
w.field("kolom2","C")
w.record("ngek","satu")
w.line(parts=[[[1,5],[5,5],[5,1],[3,3],[1,1]]])
w.save("soal5")
\end{verbatim}

\par Hasil dari script python soal 5 diatas adalah: 
\begin{figure}[!htbp]
    \centering
    \includegraphics[scale=0.7]{figures/soal5.PNG}
    \label{soal5}
\end{figure}

\section{Soal 6}
\begin{verbatim}
import shapefile
w=shapefile.Writer()
w.shapeType
w.field("kolom1","C")
w.field("kolom2","C")
w.record("ngek","satu")
w.poly(parts=[[[1,3],[5,3]]], shapeType=shapefile.POLYLINE)
w.save("soal6")
\end{verbatim}

\par Hasil dari script python soal 6 diatas adalah: 
\begin{figure}[!htbp]
    \centering
    \includegraphics[scale=0.7]{figures/soal6.PNG}
    \label{soal6}
\end{figure}

\section{Soal 7}
\begin{verbatim}
import shapefile
w=shapefile.Writer()
w.shapeType
w.field("kolom1","C")
w.field("kolom2","C")
w.record("ngek","satu")
w.poly(parts=[[[1,3],[5,3],[1,2],[5,2]]],shapeType=shapefile.POLYLIN
E)
w.save("soal7")
\end{verbatim}

\par Hasil dari script python soal 7 diatas adalah: 
\begin{figure}[!htbp]
    \centering
    \includegraphics[scale=0.7]{figures/soal7.PNG}
    \label{soal7}
\end{figure}

\section{Soal 8}
\begin{verbatim}
import shapefile
w=shapefile.Writer()
w.shapeType
w.field("kolom1","C")
w.field("kolom2","C")
w.record("ngek","satu")
w.poly(parts=[[[1,3],[5,3],[1,2],[5,2],
[1,3]]],shapeType=shapefile.POLYLINE)
w.save("soal8")
\end{verbatim}

\par Hasil dari script python soal 8 diatas adalah: 
\begin{figure}[!htbp]
    \centering
    \includegraphics[scale=0.7]{figures/soal8.PNG}
    \label{soal8}
\end{figure}

\section{Soal 9}
\begin{verbatim}
import shapefile
w=shapefile.Writer()
w.shapeType
w.field("kolom1","C")
w.field("kolom2","C")
w.record("ngek","satu")
w.record("crot","dua")
w.poly(parts=[[[1,3],[5,3], [5,2],[1,2],
[1,3]]],shapeType=shapefile.POLYLINE)
w.poly(parts=[[[1,6],[5,6], [5,9],[1,9],
[1,6]]],shapeType=shapefile.POLYLINE)
w.save("soal9")
\end{verbatim}

\par Hasil dari script python soal 9 diatas adalah: 
\begin{figure}[!htbp]
    \centering
    \includegraphics[scale=0.7]{figures/soal9.PNG}
    \label{soal9}
\end{figure}

\section{Soal 10}
\par Hasil dari script python soal 10 diatas adalah: 
\begin{figure}[!htbp]
    \centering
    \includegraphics[scale=0.7]{figures/soal10.PNG}
    \label{soal10}
\end{figure}





 